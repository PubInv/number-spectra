\documentclass[11pt]{article}
\usepackage{geometry} % see geometry.pdf on how to lay out the page. There's lots.
\usepackage{hyperref}
\usepackage{graphicx}
\usepackage{gensymb}
\usepackage[affil-it]{authblk}
\usepackage[toc,page]{appendix}
\usepackage{pifont}
\usepackage{amsmath}
\usepackage{amssymb}
\usepackage{relsize}
\usepackage{draftwatermark}
\usepackage[mathscr]{eucal}
\usepackage{amsmath, amssymb, graphics, setspace}
\usepackage{amsthm}
\usepackage[english]{babel}
\usepackage{listings}
\lstloadlanguages{Mathematica}
\usepackage{mathtools}
\DeclarePairedDelimiter{\norm}{\lVert}{\rVert}



\newcommand{\mathsym}[1]{{}}
\newcommand{\unicode}[1]{{}}

\newtheorem{exercise}{Exercise}
\newtheorem{theorem}{Theorem}
\newtheorem{corollary}{Corollary}
\newtheorem{conjecture}{Conjecture}
\newtheorem{observation}{Observation}



\SetWatermarkText{DRAFT}
\SetWatermarkScale{6}
\SetWatermarkLightness{0.95}

\DeclareMathOperator{\atantwo}{atan2}
\DeclareMathOperator{\sign}{sign}
\DeclareMathOperator{\sense}{sense}

% \geometry{letter} % or letter or a5paper or ... etc
% \geometry{landscape} % rotated page geometry

% See the ``Article customise'' template for come common customisations

\title{An Investigation into Mapping Decimal Expansions into Simple Expressions}
\author{Robert L. Read
  \thanks{read.robert@gmail.com}
   email \href{mailto:read.robert@gmail.com}{read.robert@gmail.com}
}
\author{Eric Goff
  \thanks{goffster@gmail.com}
  email \href{mailto:goffster@gmail.com}{goffster@gmail.com}
}

\affil{Founder, Public Invention, a non-profit.}

\date{\today}

%%% BEGIN DOCUMENT
\begin{document}

\maketitle

\abstract{
  }
\tableofcontents


\section{Introduction}

\section{Integer Spectrum}

Our first approach is to develop what we call the ``integer spectra'', which
are perhaps the simplest possible generation rules.  These rules
generate expressions of definite length, counted as one for each symbol
use.

Define $E_n$ to be the set of all expressions of precisely length $n$.

The generating rules are:
\begin{itemize}
\item ``1'' is a zero-ary symbol.
  \item $(! k)$ is a unary symbol operating on a expression $k$.
  \item $(+ k j)$ is a binary operation on $k$ and $j$ where $j$ is less than $k$, by the ordering of the size of the expression first and then the value
    if the sizes are the same.
  \item $(* k j)$ is similar.
  \item $(** k j)$ is has the value $k^j$.
    \item $-k$ is a unary operation.
\end{itemize}
(This of course could be thought of as a grammar.)
From this grammar, a straigforward algorithm generates $E_n$ with all smaller
classes as input.

\end{document}
